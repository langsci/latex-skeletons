\documentclass[12pt]{article}
\usepackage{graphicx}
\usepackage{tikz}
\usetikzlibrary{positioning}
\usepackage{calc}
\usepackage{xstring}
\usepackage{pst-barcode}

\hyphenpenalty=10000

% CAUTION!


%\usetikzlibrary{calc}
\usepackage{xcolor}
\definecolor{lsLightBlue}{cmyk}{0.6,0.05,0.05,0}
\definecolor{lsMidBlue}{cmyk}{0.75,0.15,0,0}
\definecolor{lsMidDarkBlue}{cmyk}{0.9,0.4,0.05,0}
\definecolor{lsDarkBlue}{cmyk}{0.9,0.5,0.15,0.3}
\definecolor{lsNightBlue}{cmyk}{1,0.47,0.22,0.68}
\definecolor{lsYellow}{cmyk}{0,0.25,1,0}
\definecolor{lsLightOrange}{cmyk}{0,0.50,1,0}
\definecolor{lsMidOrange}{cmyk}{0,0.64,1,0}
\definecolor{lsDarkOrange}{cmyk}{0,0.78,1,0}
\definecolor{lsRed}{cmyk}{0.05,1,0.8,0}
\definecolor{lsLightWine}{cmyk}{0.3,1,0.6,0}
\definecolor{lsMidWine}{cmyk}{0.54,1,0.65,0.1}
\definecolor{lsDarkWine}{cmyk}{0.58,1,0.70,0.35}
\definecolor{lsSoftGreen}{cmyk}{0.32,0.02,0.72,0}
\definecolor{lsLightGreen}{cmyk}{0.4,0,1,0}
\definecolor{lsMidGreen}{cmyk}{0.55,0,0.9,0.1}
\definecolor{lsRichGreen}{cmyk}{0.6,0,0.9,0.35}
\definecolor{lsDarkGreen1}{cmyk}{0.85,0.02,0.95,0.38}
\definecolor{lsDarkGreen2}{cmyk}{0.85,0.05,1,0.5}
\definecolor{lsNightGreen}{cmyk}{0.88,0.15,1,0.66}
\definecolor{lsLightGray}{cmyk}{0,0,0,0.17}
\definecolor{lsGuidelinesGray}{cmyk}{0,0.04,0,0.45}
 
\input{../../LSP/lsp-seriesinfo/seriesinfo.tex} 
%%%%%%%%%%%%%%%%%%%%%%%%%%%%%%%%%%%%%%%%%%%%%%%%%%%%
%%%                                              %%%
%%%                 Metadata                     %%%
%%%          fill in as appropriate              %%%
%%%                                              %%%
%%%%%%%%%%%%%%%%%%%%%%%%%%%%%%%%%%%%%%%%%%%%%%%%%%%%

\title{The Alor-Pantar languages}  %look no further, you can change those things right here.
\subtitle{History and typology}
\BackTitle{The Alor-Pantar languages} % Change if BackTitle != Title
\BackBody{The Alor-Pantar family constitutes the westernmost outlier group of Papuan (Non-Austronesian) languages. Its twenty or so languages are spoken on the islands of Alor and Pantar, located just north of Timor, in eastern Indonesia. Together with the Papuan languages of Timor, they make up the Timor-Alor-Pantar family. The languages average 5,000 speakers and are under pressure from the local Malay variety as well as the national language, Indonesian. 
 
This volume studies the internal and external linguistic history of this interesting group, and showcases some of its unique typological features, such as the preference to index the transitive patient-like argument on the verb but not the agent-like one; the extreme variety in morphological alignment patterns; the use of plural number words; the existence of quinary numeral systems; the elaborate spatial deictic systems involving an elevation component; and the great variation exhibited in their kinship systems.
 
Unlike many other Papuan languages, Alor-Pantar languages do not exhibit clause-chaining, do not have switch reference systems, never suffix subject indexes to verbs, do not mark gender, but do encode clusivity in their pronominal systems. Indeed, apart from a broadly similar head-final syntactic profile, there is little else that the Alor-Pantar languages share with Papuan languages spoken in other regions. While all of them show some traces of contact with Austronesian languages, in general, borrowing from Austronesian has not been intense, and contact with Malay and Indonesian is a relatively recent phenomenon in most of the Alor-Pantar region.}
%\dedication{Change dedication in localmetadata.tex}
%\typesetter{Change typesetter in localmetadata.tex}
%\proofreader{Change proofreaders in localmetadata.tex}
\author{Marian Klamer}

\renewcommand{\lsISBN}{978-3-944675-49-7}                     
\renewcommand{\lsSeries}{sidl} % use lowercase acronym, e.g. sidl, eotms, tgdi
\renewcommand{\lsSeriesNumber}{3} %will be assigned when the book enters the proofreading stage
\renewcommand{\lsURL}{http://langsci-press.org/catalog/book/66} % contact the coordinator for the right number

\StrLen{\subtitle}[\subtitleStrLen]


\newlength{\seitenbreite}
\newlength{\seitenhoehe}
\newlength{\spinewidth}
\newlength{\totalwidth}
%\newlength{\totalheight}
\setlength{\seitenbreite}{169.9mm}
\setlength{\seitenhoehe}{244.1mm}

\setlength{\spinewidth}{\csspine}			% Create Space Version. Value is in ./localmetadata.tex
%\setlength{\spinewidth}{\bodspine}   % BoD Version. Value is in ./localmetadata.tex

\setlength{\totalwidth}{\spinewidth+\seitenbreite+\seitenbreite}
%\setlength{\totalheight}{\seitenhoehe}
\usepackage[paperheight=\seitenhoehe, paperwidth=\totalwidth]{geometry}

	%
	% FONT MATTERS
	%
	
\usepackage{fontspec}
\newcommand{\fontpath}{../../LSP/lsp-fonts/}
\setsansfont[
	%Ligatures={TeX,Common},		% not supported by ttf
	Scale=MatchLowercase,
	Path=\fontpath,
	BoldFont = Arimo-Bold_B.ttf ,
	ItalicFont = Arimo-Italic_B.ttf ,				
	BoldItalicFont = Arimo-BoldItalic_B.ttf 		
	]{Arimo_B.ttf}
\setmonofont[
	Ligatures={TeX},Scale=MatchLowercase,Path=\fontpath,
	BoldFont = FreeMonoBold_B.otf ,
	SlantedFont = FreeMonoOblique_B.otf ,				
	BoldSlantedFont = FreeMonoBoldOblique_B.otf 		
	]{FreeMono_B.otf}
\setmainfont[
	Ligatures={TeX,Common},
	Path=\fontpath,
	PunctuationSpace=0,							
	%Numbers={OldStyle,Proportional},				% for tables use \addfontfeatures{Numbers={Monospaced,Lining}}
	Numbers={Proportional},	% normal numbers			% for tables use \addfontfeatures{Numbers={Monospaced,Lining}}
	BoldFont = LinLibertine_RZ_B.otf ,				% semi-bold
	ItalicFont = LinLibertine_RI_B.otf ,			
	BoldItalicFont = LinLibertine_RZI_B.otf 		% semi-bold
	]{LinLibertine_R_B.otf}			

  \newcommand{\lsCoverTitleFont}[1]{\sffamily\addfontfeatures{Scale=MatchUppercase}\fontsize{52pt}{16.75mm}\selectfont #1}
  \newcommand{\lsCoverSubTitleFont}{\sffamily\addfontfeatures{Scale=MatchUppercase}\fontsize{25pt}{10mm}\selectfont}
  \newcommand{\lsCoverAuthorFont}{\fontsize{25pt}{12.5mm}\selectfont}
  \newcommand{\lsCoverSeriesFont}{\sffamily\fontsize{17pt}{7.5mm}\selectfont}			% fontsize?
  \newcommand{\lsCoverSeriesHistoryFont}{\sffamily\fontsize{10pt}{5mm}\selectfont}
  \newcommand{\lsInsideFont}{}	% obsolete, see \setmainfont
  \newcommand{\lsDedicationFont}{\fontsize{15pt}{10mm}\selectfont}
  \newcommand{\lsBackTitleFont}{\sffamily\addfontfeatures{Scale=MatchUppercase}\fontsize{25pt}{10mm}\selectfont}
  \newcommand{\lsBackBodyFont}{\lsInsideFont}
  \newcommand{\lsSpineAuthorFont}{\fontsize{16pt}{14pt}\selectfont}
  \newcommand{\lsSpineTitleFont}{\sffamily\fontsize{18pt}{14pt}\selectfont}
	\newcommand{\lsCoverFontColour}{white} % Insert the Colour for Text and Logos on Cover here.
	
	%
	% END FONT MATTERS
	%
	
\begin{document}
\pagestyle{empty}
\pgfdeclarelayer{lspcls_bg} % Please make sure to never use lspcls_... PGF layers in any document
\pgfsetlayers{lspcls_bg,main}

	\begin{tikzpicture}[remember picture, overlay,bg/.style={outer sep=0}]
		\begin{pgfonlayer}{lspcls_bg} % background layer
			\node [bg, left = 7.5mm of current page.east, fill=\lsSeriesColor, minimum height=22.5cm, minimum width=15.5cm] (lspcls_bg1) {}; % Die können wir noch dynamisch bestimmen
			\node [bg, right = 7.5mm of current page.west, fill=\lsSeriesColor, minimum height=22.5cm, minimum width=15.5cm] (lspcls_bg2) {};
			\node at (current page.center) [bg, minimum height=24cm, minimum width=\spinewidth,dashed] (lspcls_bgspline) {}; % add draw option for preview mode 
		\end{pgfonlayer}
		
		% Text and Graphics Layer
		
			% Spine
			\node [above = 7.5mm of lspcls_bgspline.south] (lspcls_splinelogo) {\color{\lsSeriesColor}\includegraphics{../../LSP/lsp-logos/langsci_spinelogo_nocolor.pdf}};
			\node [above left = 15mm and 4mm of lspcls_splinelogo.north, rotate=270] (lspcls_splinetitle) {\color{\lsSeriesColor} \lsSpineAuthorFont{\author~(ed.)} \hspace{13mm} \lsSpineTitleFont{\title}}; 
			
			% Book Cover
			\node [below right = 10mm and 7.5mm of lspcls_bg1.north west, text width=130mm, align=left] (lspcls_covertitle) {\color{\lsCoverFontColour}\lsCoverTitleFont{\title\par}}; % x = 15mm - 7.5mm ; y = 17.5mm - 7.5mm
			
			\ifnum\subtitleStrLen=0  % Is there a subtitle?
				\node [below = 11.2mm of lspcls_covertitle.south, text width=130mm] {\color{\lsCoverFontColour}\lsCoverAuthorFont{Edited~by~\author}}; % If not, just print the author
				\else
				\node [below = 8mm of lspcls_covertitle.south, text width=130mm] (lspcls_coversubtitle) {\color{\lsCoverFontColour} \lsCoverSubTitleFont \subtitle}; 
				\node [below = 11.2mm of lspcls_coversubtitle.south, text width=130mm] {\color{\lsCoverFontColour}\lsCoverAuthorFont{Edited~by~\author}};
			\fi
			
			\node [below right = 197.5mm and 117.1mm of lspcls_bg1.north west] {\color{\lsCoverFontColour}\includegraphics{../../LSP/lsp-logos/langsci_logo_nocolor.pdf}};
			\node [below right = 201.5mm and -.1mm of lspcls_bg1.north west, rectangle, fill=white, minimum size=17pt] (lspcls_square) {}; % 2
			\node [right = 3mm of lspcls_square.east, text width=95mm] {\color{\lsCoverFontColour}\lsCoverSeriesFont{\lsSeriesTitle}};
			
			% Book Back Cover
			\node [below right = 16.5mm and 7.5mm of lspcls_bg2.north west, text width=11.5cm] (lspcls_backtitle) {\color{\lsCoverFontColour}\lsBackTitleFont{\BackTitle\par}};
			\node [below = 10mm of lspcls_backtitle, text width=11.5cm, align=justify] {\color{\lsCoverFontColour}\lsBackBodyFont{\parindent=15pt\BackBody}};
			%\node [below right = 192.5mm and 97.5mm of lspcls_bg2.north west] {\color{\lsCoverFontColour}ISBN \lsBackBodyFont{\lsISBN}};
			\node [below right = 192.5mm and 97.5mm of lspcls_bg2.north west, text width=4cm] {
				\colorbox{white}{
					\begin{pspicture}(0,0)(4.1,1in)
					\psbarcode[transx=.4,transy=.3]{\lsISBN}{includetext height=.7}{isbn}\end{pspicture}}};
			
			
	\end{tikzpicture}
\end{document}