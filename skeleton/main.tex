%%%%%%%%%%%%%%%%%%%%%%%%%%%%%%%%%%%%%%%%%%%%%%%%%%%%
%%%     Language Science Press Master File       %%%
%%%         follow the instructions below        %%%
%%%%%%%%%%%%%%%%%%%%%%%%%%%%%%%%%%%%%%%%%%%%%%%%%%%%

% Everything following a % is ignored
% Some lines start with %. Remove the % to include them

\documentclass[output=book,
  colorlinks,citecolor=brown,
%  draft,draftmode,
%  showindex,
%  nobabel,
%  booklanguage=german,
		  ]{langscibook}

%%%%%%%%%%%%%%%%%%%%%%%%%%%%%%%%%%%%%%%%%%%%%%%%%%%%
%%%          additional packages                 %%%
%%%%%%%%%%%%%%%%%%%%%%%%%%%%%%%%%%%%%%%%%%%%%%%%%%%%

%%%%%%%%%%%%%%%%%%%%%%%%%%%%%%%%%%%%%%%%%%%%%%%%%%%%
%%%                                              %%%
%%%                 Metadata                     %%%
%%%          fill in as appropriate              %%%
%%%                                              %%%
%%%%%%%%%%%%%%%%%%%%%%%%%%%%%%%%%%%%%%%%%%%%%%%%%%%%

\title{The Alor-Pantar languages}  %look no further, you can change those things right here.
\subtitle{History and typology}
\BackTitle{The Alor-Pantar languages} % Change if BackTitle != Title
\BackBody{The Alor-Pantar family constitutes the westernmost outlier group of Papuan (Non-Austronesian) languages. Its twenty or so languages are spoken on the islands of Alor and Pantar, located just north of Timor, in eastern Indonesia. Together with the Papuan languages of Timor, they make up the Timor-Alor-Pantar family. The languages average 5,000 speakers and are under pressure from the local Malay variety as well as the national language, Indonesian. 
 
This volume studies the internal and external linguistic history of this interesting group, and showcases some of its unique typological features, such as the preference to index the transitive patient-like argument on the verb but not the agent-like one; the extreme variety in morphological alignment patterns; the use of plural number words; the existence of quinary numeral systems; the elaborate spatial deictic systems involving an elevation component; and the great variation exhibited in their kinship systems.
 
Unlike many other Papuan languages, Alor-Pantar languages do not exhibit clause-chaining, do not have switch reference systems, never suffix subject indexes to verbs, do not mark gender, but do encode clusivity in their pronominal systems. Indeed, apart from a broadly similar head-final syntactic profile, there is little else that the Alor-Pantar languages share with Papuan languages spoken in other regions. While all of them show some traces of contact with Austronesian languages, in general, borrowing from Austronesian has not been intense, and contact with Malay and Indonesian is a relatively recent phenomenon in most of the Alor-Pantar region.}
%\dedication{Change dedication in localmetadata.tex}
%\typesetter{Change typesetter in localmetadata.tex}
%\proofreader{Change proofreaders in localmetadata.tex}
\author{Marian Klamer}

\renewcommand{\lsISBN}{978-3-944675-49-7}                     
\renewcommand{\lsSeries}{sidl} % use lowercase acronym, e.g. sidl, eotms, tgdi
\renewcommand{\lsSeriesNumber}{3} %will be assigned when the book enters the proofreading stage
\renewcommand{\lsURL}{http://langsci-press.org/catalog/book/66} % contact the coordinator for the right number
\usepackage{tabularx,multicol}
\usepackage{url}
\urlstyle{same}


\usepackage{langsci-branding}
\usepackage{langsci-optional}
\usepackage{langsci-lgr}
\usepackage{langsci-gb4e}

\input{localhyphenation.tex}
\addbibresource{localbibliography.bib}

\usepackage{todonotes}

%%%%%%%%%%%%%%%%%%%%%%%%%%%%%%%%%%%%%%%%%%%%%%%%%%%%
%%%             Frontmatter                      %%%
%%%%%%%%%%%%%%%%%%%%%%%%%%%%%%%%%%%%%%%%%%%%%%%%%%%%
\usepackage[linguistics]{forest}
\usetikzlibrary{shapes,fit,positioning}
\begin{document}

\begin{forest}
 [IP,name=IP
    [~]
    [I$'$
        [VP
            [~]
            [V$^0$
                [o, name=o1]
            ]
        ]
        [I$^0$
            [T$^0$
                [o, name=o2]
            ]
            [AGR$^0$
                [o, name=o3]
            ]
        ]
    ]
 ]
\node(listA)[above=2cm of IP,xshift=-2cm]{List A: };
\node(kauf)[right=0mm of listA,draw,dashed,rectangle,rounded corners,inner sep=1mm]{$\sqrt{\text{KAUF}}$};
\node(rest)[right=0mm of kauf]{, $\sqrt{\text{LAUF}}$,$\sqrt{\text{HUND}}$,$\sqrt{\text{GUT}}$, \ldots\\};
\node(tensepast)[below=2mm of kauf,draw,dashed,rectangle,rounded corners,inner sep=1mm]{\footnotesize [\textsc{tense}:\textsc{past}]};
\node(person12)[right=0mm of tensepast]{\footnotesize , [\textsc{person}:1], [\textsc{person}:2], };
\node(person3numpl)[right=0mm of person12,draw,dashed,rectangle,rounded corners,inner sep=1mm]{\footnotesize [\textsc{person}:3], [\textsc{num}:\textsc{pl}]};
\node(genf)[right=0mm of person3numpl]{\footnotesize , [\textsc{gen}:\textsc{f}]};
%
\node(vkaufpast3pl)[below = 9cm of kauf]{$\sqrt{\text{KAUF}}$-\textsc{past}-3-\textsc{pl}};
\node(listB)[below = 3cm of vkaufpast3pl,xshift=-2cm]{List B:};
\node(kaOf)[right=0mm of listB]{/kaOf/,};
\node(past-t@)[right=0mm of kaOf]{[\textsc{past}] $\leftrightarrow$ /t@/,};
\node(2pl-t)[right=0mm of past-t@,text width=2.5cm,text align=center]{[2\textsc{pl}] $\leftrightarrow$ /t/, \dots};
\node(listC)[right=0mm of 2pl-t]{List C: img img img};
\node(remainder)[below=0mm of 2pl-t,text width=2cm]{[\textsc{pl}] $\leftrightarrow$ /n/\\
    {}[\textsc{2}] $\leftrightarrow$ /st/\\
    {}[] $\leftrightarrow$ /$\empty$/};
\draw[->](o2)--(vkaufpast3pl);
\draw(kaOf) -- node[above,sloped] {\footnotesize match} (vkaufpast3pl);
\draw(past-t@)--node[above,sloped] {\footnotesize match} (vkaufpast3pl);
\draw(2pl-t.north)--node[above,sloped] {\footnotesize 1st trial no match}(vkaufpast3pl);
\draw(remainder.north)--node[below,sloped] {\footnotesize  2nd trial match}(vkaufpast3pl);
%
\path[dashed](o1)edge [bend left=70,in=40] (kauf.north);
\path[dashed](o2)edge [bend left=70] (tensepast);
\draw[dashed](o3.east)edge [bend right=40](person3numpl);
\path[gray](listC.east) edge [out=30,in=30] (kauf.north);
\end{forest}


%\maketitle
%\frontmatter

\newcommand{\appref}[1]{Appendix \ref{#1}}
\newcommand{\fnref}[1]{Footnote \ref{#1}} 

\newenvironment{langscibars}{\begin{axis}[ybar,xtick=data, xticklabels from table={\mydata}{pos}, 
        width  = \textwidth,
	height = .3\textheight,
    	nodes near coords, 
	xtick=data,
	x tick label style={},  
	ymin=0,
	cycle list name=langscicolors
        ]}{\end{axis}}
        
\newcommand{\langscibar}[1]{\addplot+ table [x=i, y=#1] {\mydata};\addlegendentry{#1};}

\newcommand{\langscidata}[1]{\pgfplotstableread{#1}\mydata;}


\makeatletter
\let\theauthor\@author %store in more accessible format
\makeatother 
 
\newcommand{\lsFirstAuthorFullName}{}%temporary, will be overwritten
\newcommand{\lsFirstAuthorFirstName}{}%temporary, will be overwritten
\newcommand{\lsFirstAuthorLastName}{}%temporary, will be overwritten
\newcommand{\lsFirstAuthorString}{\lsFirstAuthorLastName, \lsFirstAuthorFirstName} %can be customized in localmetadata.tex
\newcommand{\lsNonFirstAuthorsString}{}  %default, will be overwritten iff more than one author
\newcommand{\lsImpressionCitationAuthor}{\lsFirstAuthorString \lsNonFirstAuthorsString}


\renewcommand{\and}{NONLASTAND} %expand for easier checking. Might need to be undone later on
\renewcommand{\lastand}{LASTAND} %expand for easier checking

\IfSubStr{\theauthor}{NONLASTAND}{%2+authors
  \renewcommand{\lsFirstAuthorFullName}{\StrBefore{\theauthor}{\and }}
  \renewcommand{\lsFirstAuthorFirstName}{\StrBefore{\theauthor}{ }} 
  \renewcommand{\lsFirstAuthorLastName}{\StrBetween{\theauthor}{ }{\and }}
  \renewcommand{\lsNonFirstAuthorsString}{\and\StrBehind{\theauthor}{\and }} 
  }{%else
    \IfSubStr{\theauthor}{LASTAND}{%less than two authors, more than one
    \renewcommand{\lsFirstAuthorFullName}{\StrBefore{\theauthor}{\lastand }}
    \renewcommand{\lsFirstAuthorFirstName}{\StrBefore{\theauthor}{ }}
    \renewcommand{\lsFirstAuthorLastName}{\StrBetween{\theauthor}{ }{\lastand }}
    \renewcommand{\lsNonFirstAuthorsString}{\lastand\StrBehind{\theauthor}{\lastand }} 
    }{%else exactly one author
      \renewcommand{\lsFirstAuthorFirstName}{\StrBefore{\theauthor}{ }}
      \renewcommand{\lsFirstAuthorLastName}{\StrBehind{\theauthor}{ }}
      }
    }   

\maketitle
\frontmatter

% \currentpdfbookmark{Contents}{name} % adds a PDF bookmark
{\sloppy\tableofcontents}
%  \addchap{Preface}
\begin{refsection}
This preface has no abstract and no authors, but cites \citet{Nordhoff2018} nevertheless.
%content goes here

\printbibliography[heading=subbibliography]
\end{refsection}


%  \include{chapters/acknowledgments}
%  \addchap{\lsAbbreviationsTitle}
% \addchap{Abbreviations and symbols}

\begin{tabularx}{.45\textwidth}{lQ}
... & \\
... & \\
\end{tabularx}
\begin{tabularx}{.45\textwidth}{lQ}
... & \\
... & \\
\end{tabularx}

\mainmatter

%%%%%%%%%%%%%%%%%%%%%%%%%%%%%%%%%%%%%%%%%%%%%%%%%%%%
%%%             Chapters                         %%%
%%%%%%%%%%%%%%%%%%%%%%%%%%%%%%%%%%%%%%%%%%%%%%%%%%%%

% \documentclass[output=paper]{LSP/langsci}
\author{John Doe from Chapters/01.tex}
\title{Change {T}itle in chapters/01.tex}
\epigram{Change epigram in chapters/01.tex or remove it there }
\abstract{Change the  abstract in chapters/01.tex \lipsum[1]}
\maketitle

\begin{document}

\section{Introduction} 
 
 \citet{Chomsky1957}
\ea\label{ex:1:descartes}
\langinfo{Latin}{}{personal knowledge}\\
\gll cogit-o ergo sum \\
     think-1{\sg}.{\prs}.{\ind} hence exist.1{\sg}.{\prs}.{\ind}\\
\glt `I think therefore I am'
\z

\begin{table}
\caption{Frequencies of word classes}
\label{tab:1:frequencies}
 \begin{tabular}{lllll} % add l for every additional column or remove as necessary
  \lsptoprule
            & nouns & verbs & adjectives & adverbs\\ %table header
  \midrule
  absolute  &   12 &    34  &    23     & 13\\
  relative  &   3.1 &   8.9 &    5.7    & 3.2\\
  \lspbottomrule
 \end{tabular}
\end{table}

\isi{prolegomena}
 


\printbibliography[heading=subbibliography,notkeyword=this]

\end{document}  %add a percentage sign in front of the line to exclude this chapter from book

% copy the lines above and adapt as necessary

%%%%%%%%%%%%%%%%%%%%%%%%%%%%%%%%%%%%%%%%%%%%%%%%%%%
%%             Backmatter                       %%%
%%%%%%%%%%%%%%%%%%%%%%%%%%%%%%%%%%%%%%%%%%%%%%%%%%%

% \is{some term| see {some other term}}
\il{some language| see {some other language}}
\issa{some term with pages}{some other term also of interest}
\ilsa{some language with pages}{some other lect also of interest}
% % There is normally no need to change the backmatter section
\phantomsection 
\addcontentsline{toc}{chapter}{Index} 
\addcontentsline{toc}{section}{\lsNameIndexTitle}
\ohead{\lsNameIndexTitle}
\printindex 
\cleardoublepage
  
\phantomsection 
\addcontentsline{toc}{section}{\lsLanguageIndexTitle}
\ohead{\lsLanguageIndexTitle} 
\printindex[lan] 
\cleardoublepage
  
\phantomsection 
\addcontentsline{toc}{section}{\lsSubjectIndexTitle}
\ohead{\lsSubjectIndexTitle} 
\printindex[sbj]
\ohead{} 

\end{document}

% you can create your book by running
% xelatex main.tex
