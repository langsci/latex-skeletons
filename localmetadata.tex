\title{Einführung in die grammatische Beschreibung des Deutschen}
\subtitle{}
\BackTitle{Einführung in die grammatische Beschreibung des Deutschen}
\BackBody{\begin{sloppypar}
\textit{Einführung in die grammatische Beschreibung des Deutschen} ist eine Einführung in die deskriptive Grammatik am Beispiel des gegenwärtigen Deutschen in den Bereichen Phonetik, Phonologie, Morphologie, Syntax und Graphematik.
Das Buch ist für jeden geeignet, der sich für die Grammatik des Deutschen interessiert, vor allem aber für Studierende der Germanistik bzw.\ Deutschen Philologie.
Im Vordergrund steht die Vermittlung grammatischer Erkenntnisprozesse und Argumentationsweisen auf Basis konkreten sprachlichen Materials.
Es wird kein spezieller theoretischer Rahmen angenommen, aber nach der Lektüre sollten Leser in der Lage sein, sowohl deskriptiv ausgerichtete Forschungsartikel als auch theorienahe Einführungen lesen zu können.
Trotz seiner Länge ist das Buch für den Unterricht in BA-Studiengängen geeignet, da grundlegende und fortgeschrittene Anteile getrennt werden und die fünf Teile des Buches auch einzeln verwendet werden können.
Das Buch enthält zahlreiche Übungsaufgaben, die im Anhang gelöst werden.

\vspace{1\baselineskip}

\textbf{Roland Schäfer} studierte Sprachwissenschaft und Japanologie an der Philipps-Universität Marburg.
Er war wissenschaftlicher Mitarbeiter an der Georg-August Universität Göttingen und der Freien Universität Berlin.
Er promovierte 2008 an der Georg-August Universität Göttingen mit einer theoretischen Arbeit zur Syntax-Semantik-Schnittstelle.
Seine aktuellen Forschungsschwerpunkte sind die korpusbasierte Morphosyntax und Graphematik des Deutschen und anderer germanischer Sprachen sowie die Erstellung sehr großer Korpora aus Webdaten.
Seit 2015 leitet er das DFG-finanzierte Projekt \textit{Linguistische Web-Charakterisierung und Webkorpuserstellung} an der Freien Universität Berlin.
Er hat langjährige Lehrerfahrung in deutscher und englischer Sprachwissenschaft sowie theoretischer Sprachwissenschaft und Computerlinguistik.
\end{sloppypar}}
\dedication{Für Mausi und so.}
\typesetter{Roland Schäfer}
\proofreader{Thea Dittrich}
\author{Roland Schäfer}
\newlength{\csspine} 
\newlength{\bodspine}
\setlength{\csspine}{27.0559784mm} % Please calculate: Total Page Number (excluding cover, usually (Total Page - 3)) * 0.0572008 mm
\setlength{\bodspine}{40mm} % Please calculate: Total Page Number (excluding cover) * BODFACTOR + BODABS
\definecolor{lsYellow}{cmyk}{0,0.25,1,0}
\newcommand{\lsSeriesColor}{lsYellow}
\newcommand{\lsSeriesTitle}{Textbooks in Language Sciences~2}
\renewcommand{\lsISBN}{978-3-944675-53-4}                     
\renewcommand{\lsSeries}{tbls} % use lowercase acronym, e.g. sidl, eotms, tgdi
\renewcommand{\lsSeriesNumber}{2} %will be assigned when the book enters the proofreading stage
\renewcommand{\lsURL}{http://langsci-press.org/catalog/book/46} % contact the coordinator for the right number